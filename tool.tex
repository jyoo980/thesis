%% The following is a directive for TeXShop to indicate the main file
%%!TEX root = diss.tex

\chapter{The \toolname{} Plugin}
\label{ch:Tool}

\begin{epigraph}
  \emph{
    Deez nuts
    } ---~James Yoo
\end{epigraph}

\noindent \toolname{} is an open-source\footnote{\url{https://github.com/jyoo980/reach-hover}}
plugin developed on the JetBrains IntelliJ Platform.
We describe how a developer might use \toolname{} in
section \ref{sec:UsingReachHover}, detail its architecture and implementation 
in section \ref{sec:Architecture}, and \dots

\section{Using \toolname{}}
\label{sec:UsingReachHover}

\subsection{Invocation Interaction}
\label{subsection:InvocationInteraction}

\todo{Talk about hovering.}

\subsection{Result Visualization}
\label{subsection:ResultVisualization}

\todo{Talk about the popup window.}

\section{Design \& Implementation}
\label{sec:Architecture}

\toolname{} is implemented in the Kotlin programming language, and consumes
\acp{API} written in Java that are exposed by the IntelliJ Platform.
The mechanism that reports the viability of the analysis that \toolname{}
exploits to produce a dataflow trace for reachability analysis is 
language-agnostic.
As a result, \toolname{} does not depend on any language-specific syntax to 
determine whether a reachability analysis is viable; it analyzes a genericized 
program syntax tree that is provided by \class{UAST} \dots

\endinput
