%% The following is a directive for TeXShop to indicate the main file
%%!TEX root = diss.tex

\chapter{The \toolname{} Plugin}
\label{ch:Tool}

% TODO: typeset the following nicely.
% \begin{epigraph}
%   \emph{What information consumes is rather obvious: it 
%     consumes the attention of its recipients.
%     Hence a wealth of information creates a poverty of attention, and a need 
%     to allocate that attention efficiently among the overabundance 
%     of information sources that might consume it.}
%      ---~Herbert A. Simon
% \end{epigraph}

\noindent \toolname{} is an open-source\footnote{\url{https://github.com/jyoo980/reach-hover}}
plugin developed on the JetBrains IntelliJ Platform.
Section \ref{sec:DesignMeth} discusses the design methodology for \toolname{},
derived from the data presented in Chapter \ref{ch:Survey}.
In section \ref{sec:UsingReachHover}, we describe how a user might use
\toolname{} in a software development scenario, with a particular focus on
its mechanism of invocation and how it visualizes data to a user.
The implementation of \toolname{} is detailed in section
\ref{sec:Impl}.

\section{Design Methodology}
\label{sec:DesignMeth}

\noindent To inform the design of \toolname{}, we analyzed the data from a 
survey that sought to identify the type and frequency of reachability questions 
that are asked by software developers in practice.

Figure \ref{fig:SurveyResults} shows that Development Questions 1, 2, 5, and 9
were the most frequently asked among our respondents.
Using Table \ref{Table:SurveyTags} to map these questions to their associated
tags, we found that $Q_{find}$ was the most frequent type of question asked out
of the top four ranked by our respondents.
Specifically, these questions were:

\begin{itemize}
  \item[] \textbf{Development Question 1}:\\ \textit{Where did a value come from,
  and/or how was it formed?}
  \item[] \textbf{Development Question 2}:\\ \textit{Given some data, which
  parts of it are modified downstream?}
\end{itemize}

\par We noted that these questions are thematically related. They are instantiations 
of the \textit{find} category of reachability question, and they can also be
answered by traversing backward and forward in a data-flow trace, respectively.
Consequently, we decided to build \toolname{} with a particular focus on
supporting these two questions for the sake of thematic coherence.

\par Since information provided from a data-flow trace could be used to answer
the questions we attempt to support with \toolname{}, we investigated existing
data-flow analysis tools to obtain an understanding of the current landscape
of analysis tooling.
The IntelliJ IDEA \ac{IDE} exposes forward and backward data-flow analysis
capabilities to users.
Figure \ref{fig:IntelliJDataflow} shows how a user might invoke a backward 
data-flow analysis, while the result is shown in
Figure \ref{fig:IntelliJDataflowResult}.
We found these pre-existing data-flow capabilities to be a good basis for the 
development of \toolname{} due to its ease of extension via the JetBrains 
IntelliJ Platform \ac{SDK}.

\par Next, we began to identify pain points that users might face as they
invoked the data-flow analysis tool.
We discovered that discovering the tool itself might be difficult for
users; the action to invoke the tool was a three-click action.
Additionally, the action was hidden behind a context menu that did not appear
to have a strong information scent that guided users to discovering and invoking
the tool from a very broad list of options (\ie data-flow-specific analyses
are accessed via a general ``Analyze" menu element).

\todo{Talk about the lack of context in the result vis.}

\section{Using \toolname{}}
\label{sec:UsingReachHover}

\subsection{Invocation Interaction}
\label{subsection:InvocationInteraction}

\todo{Talk about hovering.}

\subsection{Result Visualization}
\label{subsection:ResultVisualization}

\todo{Talk about the popup window.}

\section{Implementation}
\label{sec:Impl}

\toolname{} is implemented in the Kotlin programming language, and consumes
\acp{API} written in Java that are exposed by the IntelliJ Platform.
The mechanism that reports the viability of the analysis that \toolname{}
exploits to produce a dataflow trace for reachability analysis is 
language-agnostic.
As a result, \toolname{} does not depend on any language-specific syntax to 
determine whether a reachability analysis is viable; it analyzes a genericized 
program syntax tree that is provided by \class{UAST} \dots

\endinput
